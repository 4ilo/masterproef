In deze thesis wordt de positie van een mobiele robot gevolgd in de logistieke gangen van een ziekenhuis door gebruik te maken van een RGB camera en een semantische kaart.

In deze thesis is er een volledige pipeline opgesteld die op basis van een input stream van RGB beelden opvallende kenmerken
gaat zoeken in de ruimte zoals lampen, rookmelders en brandblussers.
In combinatie met perspectiefpunt detectie en informatie over deze gedetecteerde objecten op de semantische kaart zal er een
schatting gemaakt worden van de actuele locatie van de robot.

We hebben onderzocht welke object detector en welke perspectiefpunt detectiemethode de beste resultaten oplevert, en op basis van
deze gegevens de volledige pipeline ge\"{i}mplementeerd in python.


\pagebreak
\chapter*{Abstract}

In this thesis we track the position of a mobile robot inside the logistic hallways of a hospital by using a RGB camera and a semantic map.

We created a whole pipeline to detect notable features in the corridors such as lamps, smoke detectors and fire extinguishers bases on a stream of RGB images.
With a combination of this object detector, a vanishing point detector and information from the semantic map about these objects we try to estimate
the current location of the robot.

We researched which object detector and vanishing point detection method delivered the best results, and implemented the processing pipeline
in python based on this information.