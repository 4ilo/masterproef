%%%%%%%%%%%%%%%%%%%%%%%%%%%%%%%%%%%%%%%%%%%%%%%%%%%%%%%%%%%%%%%%%%% 
%                                                                 %
%                          Uitwerking                             %
%                                                                 %
%%%%%%%%%%%%%%%%%%%%%%%%%%%%%%%%%%%%%%%%%%%%%%%%%%%%%%%%%%%%%%%%%%% 

\chapter{Uitwerking}

\begin{figure}[!h]
    \centering
    \includegraphics[width=\linewidth]{uitwerking/pipeline1.png}
    \caption{Overzicht van het programma}
    \label{fig:pipeline1}
\end{figure}

Het doel van deze masterproef is om op basis van 1 \gls{rgb} camera en een semantische kaart een robot autonoom te laten navigeren in de logistieke gangen van een ziekenhuis.
Om dit te implementeren, is er een pipeline opgesteld. Deze pipeline is onderverdeeld in een aantal stappen die in de volgende hoofdstukken uitgebreid besproken worden.

\section{Object detectie}

Zoals besproken zal worden in hoofdstuk~\ref{sec:sem_kaart}, bevat de semantische kaart van het ziekenhuis een beschrijving van objecten/features die aanwezig zijn in de gangen.
Deze objecten zijn bijvoorbeeld:
\begin{itemize}
    \item Lampen;
    \item Brandblussers;
    \item Pictogrammen;
    \item Rookdetectors;
    \item Deurklinken.
\end{itemize}

De gebruikte dataset zijn 2 video's opgenomen vanop een robot met de de waylens horizon\footnote{https://waylens.com/horizon/techSpecs/} \gls{rgb} camera.
In dit beeldmateriaal zijn er een aantal van bovenstaande objecten zichtbaar, en kan er een detector gemaakt worden om deze features te herkennen.

De gekozen object detector is YOLOv2\cite{yolov22016}. Dit is een zeer veel gebruikt object detectie/classificatie convolutioneel neuraal netwerk dat toelaat om
objecten van een geleerde klasse te herkennen en lokaliseren in nieuwe beelden.

\subsection{Annotatie beeldmateriaal}
Om een object-detector te maken gebaseerd op een \gls{cnn} moet al het beeldmateriaal geannoteerd worden door bounding-boxes te teken rondom de kenmerkende objecten.
Elk van deze bounding-boxes moet ook voorzien worden van een bijhorend label. In tabel~\ref{tab:annotaties} is een opsomming gemaakt van alle annotaties in de training en validatie set.

\begin{table}[h]
    \caption{Annotaties in training en validatie set}\label{tab:annotaties}
    \begin{tabular}{l | l | l | l | l | l | l}
        & Aantal frames & Rookmelder & Deurklink & Pictogram & TL-lamp & Totaal \\ \hline
        Trainings set & 899 & 1016 & 147 & 340 & 5260 & 6763 \\
        Validatie set & 711 & 1130 & 180 & 408 & 992 & 2710 \\
    \end{tabular}
\end{table}

Voor het aanduiden van de annotaties is er gebruik gemaakt van de OpenCV \gls{cvat}\footnote{https://github.com/opencv/cvat}.
De output van de \gls{cvat} tool is een XML-bestand met voor elke afbeelding de co\"{o}rdinaten van de objecten. Een voorbeeld output voor 1 enkele afbeelding is te zien in listing~\ref{lst:cvat_xml}.

\lstset{language=XML, caption={Voorbeeld CVAT output}, label={lst:cvat_xml}}
   \begin{lstlisting}[basicstyle=\small]
<image id="24" name="hospital_corridors_025.png" width="1280" height="720">
   <box label="door_handle" xtl="863" ytl="237" xbr="881" ybr="248"/>
   <box label="light" xtl="584" ytl="254" xbr="608" ybr="265"/>
   <box label="light" xtl="435" ytl="321" xbr="448" ybr="329"/>
   <box label="exit_sign" xtl="590" ytl="298" xbr="604" ybr="307"/>
</image>
   \end{lstlisting}

De trainingsdata die aan het \gls{yolo} netwerk geleverd moet worden is in een volledig ander formaat dan het verkregen XML bestand, daarom is er een conversiescript geschreven om dit om te zetten naar het \gls{yolo} formaat.
Het conversiescript bevind zich in bijlage~\ref{bij:convert}.
Het formaat dat \gls{yolo} verwacht is per input afbeelding een .txt bestand met daarin per lijn een bounding box. Het \gls{yolo} formaat is beschreven in~\ref{eq:yolo} waarbij $<x>$ en $<y>$ het centerpunt van een bounding box zijn.
Alle waarden zijn genormaliseerd, dit wil zeggen dat om het absolute $(x, y)$ co\"{o}rdinaat te bekomen, de waardes vermenigvuldigd moeten worden met de breedte en de hoogte zoals ge\"{i}llustreerd in~\ref{eq:yolo_abs}.

\begin{equation} \label{eq:yolo}
  <x> <y> <width> <height>
\end{equation}

\begin{equation} \label{eq:yolo_abs}
    \begin{split}
        x_{abs} &= <x> *  \mathrm{image\_width} \\
        y_{abs} &= <y> * \mathrm{image\_height}
    \end{split}
\end{equation}

\subsection{Training YOLOv2}

Voor training en later ook detectie met YOLOv2 is er gebruikt gemaakt van een aangepaste versie van de darknet\footnote{https://github.com/AlexeyAB/darknet} implementatie in C.
We zijn vertrokken van het yolo-voc.2.0.cfg configuratie bestand waar we het aantal te detecteren klassen hebben aangepast naar 4 (light, smoke\_detector, exit\_sign en door\_handle), en het aantal filters naar 25.
Dit nieuw configuratie bestand specificeert de netwerk layout, deze layout is een YOLOv2 netwerk waarbij nu de laatste lagen aangepast zijn om slechts 4 objecten te classificeren.

Het trainen gebeurd nu op basis van de traininsset zoals weergegeven in tabel~\ref{tab:annotaties}. De performantie van het netwerk zal besproken worden in hoofdstuk~\ref{chap:resultaten}.
\section{Perspectiefpunt detectie}

\section{Omgevingsmeting}

\section{Semantische kaart}\label{sec:sem_kaart}

\section{Lokalisatie}