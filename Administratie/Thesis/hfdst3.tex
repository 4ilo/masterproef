%%%%%%%%%%%%%%%%%%%%%%%%%%%%%%%%%%%%%%%%%%%%%%%%%%%%%%%%%%%%%%%%%%% 
%                                                                 %
%                          Uitwerking                             %
%                                                                 %
%%%%%%%%%%%%%%%%%%%%%%%%%%%%%%%%%%%%%%%%%%%%%%%%%%%%%%%%%%%%%%%%%%% 

\chapter{Uitwerking}

\begin{figure}[!h]
    \centering
    \includegraphics[width=\linewidth]{uitwerking/pipeline1.png}
    \caption{Overzicht van het programma}
    \label{fig:pipeline1}
\end{figure}

Het doel van deze masterproef is om op basis van 1 \gls{rgb} camera en een semantische kaart een robot autonoom te laten navigeren in de logistieke gangen van een ziekenhuis.
Om dit te implementeren, is er een pipeline opgesteld. Deze pipeline is onderverdeeld in een aantal stappen die in de volgende hoofdstukken uitgebreid besproken worden.


\section{Object detectie}\label{sec:object_detectie}

Zoals besproken zal worden in hoofdstuk~\ref{sec:sem_kaart}, bevat de semantische kaart van het ziekenhuis een beschrijving van objecten/features die aanwezig zijn in de gangen.
Deze objecten zijn bijvoorbeeld:
\begin{itemize}
    \item Lampen;
    \item Brandblussers;
    \item Pictogrammen;
    \item Rookdetectors;
    \item Deurklinken.
\end{itemize}

De gebruikte dataset zijn 2 video's opgenomen vanop een robot met de de waylens horizon\footnote{https://waylens.com/horizon/techSpecs/} \gls{rgb} camera.
In dit beeldmateriaal zijn er een aantal van bovenstaande objecten zichtbaar, en kan er een detector gemaakt worden om deze features te herkennen.

De gekozen object detector is YOLOv2\cite{yolov22016}. Dit is een zeer veel gebruikt object detectie/classificatie convolutioneel neuraal netwerk dat toelaat om
objecten van een geleerde klasse te herkennen en lokaliseren in nieuwe beelden.


\subsection{Annotatie beeldmateriaal}
Om een object-detector te maken gebaseerd op een \gls{cnn} moet al het beeldmateriaal geannoteerd worden door bounding-boxes te teken rondom de kenmerkende objecten.
Elk van deze bounding-boxes moet ook voorzien worden van een bijhorend label. In tabel~\ref{tab:annotaties} is een opsomming gemaakt van alle annotaties in de training en validatie set.

\begin{table}[h]
    \caption{Annotaties in training en validatie set}\label{tab:annotaties}
    \begin{tabular}{l | l | l | l | l | l | l}
        & Aantal frames & Rookmelder & Deurklink & Pictogram & TL-lamp & Totaal \\ \hline
        Trainings set & 899 & 1016 & 147 & 340 & 5260 & 6763 \\
        Validatie set & 711 & 1130 & 180 & 408 & 992 & 2710 \\
    \end{tabular}
\end{table}

Voor het aanduiden van de annotaties is er gebruik gemaakt van de OpenCV \gls{cvat}\footnote{https://github.com/opencv/cvat}.
De output van de \gls{cvat} tool is een XML-bestand met voor elke afbeelding de co\"{o}rdinaten van de objecten. Een voorbeeld output voor 1 enkele afbeelding is te zien in listing~\ref{lst:cvat_xml}.

\lstset{language=XML, caption={Voorbeeld CVAT output}, label={lst:cvat_xml}}
   \begin{lstlisting}[basicstyle=\small]
<image id="24" name="hospital_corridors_025.png" width="1280" height="720">
   <box label="door_handle" xtl="863" ytl="237" xbr="881" ybr="248"/>
   <box label="light" xtl="584" ytl="254" xbr="608" ybr="265"/>
   <box label="light" xtl="435" ytl="321" xbr="448" ybr="329"/>
   <box label="exit_sign" xtl="590" ytl="298" xbr="604" ybr="307"/>
</image>
   \end{lstlisting}

De trainingsdata die aan het \gls{yolo} netwerk geleverd moet worden is in een volledig ander formaat dan het verkregen XML bestand, daarom is er een conversiescript geschreven om dit om te zetten naar het \gls{yolo} formaat.
Het conversiescript bevind zich in bijlage~\ref{ch:bij_convert}.
Het formaat dat \gls{yolo} verwacht is per input afbeelding een .txt bestand met daarin per lijn een bounding box. Het \gls{yolo} formaat is beschreven in~\ref{eq:yolo} waarbij $<x>$ en $<y>$ het centerpunt van een bounding box zijn.
Alle waarden zijn genormaliseerd, dit wil zeggen dat om het absolute $(x, y)$ co\"{o}rdinaat te bekomen, de waardes vermenigvuldigd moeten worden met de breedte en de hoogte zoals ge\"{i}llustreerd in~\ref{eq:yolo_abs}.

\begin{equation} \label{eq:yolo}
  <x> <y> <width> <height>
\end{equation}

\begin{equation} \label{eq:yolo_abs}
    \begin{split}
        x_{abs} &= <x> *  \mathrm{image\_width} \\
        y_{abs} &= <y> * \mathrm{image\_height}
    \end{split}
\end{equation}


\subsection{Training YOLOv2}

Voor training en later ook detectie met YOLOv2 is er gebruikt gemaakt van een aangepaste versie van de darknet\footnote{https://github.com/AlexeyAB/darknet} implementatie in C.
We zijn vertrokken van het yolo-voc.2.0.cfg configuratie bestand waar we het aantal te detecteren klassen hebben aangepast naar 4 (light, smoke\_detector, exit\_sign en door\_handle), en het aantal filters naar 25.
Dit nieuw configuratie bestand specificeert de netwerk layout, deze layout is een YOLOv2 netwerk waarbij nu de laatste lagen aangepast zijn om slechts 4 objecten te classificeren.

Het trainen gebeurd nu op basis van de traininsset zoals weergegeven in tabel~\ref{tab:annotaties}. De performantie van het netwerk zal besproken worden in hoofdstuk~\ref{ch:resultaten}.


\section{Perspectiefpunt detectie}

Ik hoofdstuk~\ref{sec:omgevingsmeting} zullen we bespreken hoe het lokaliseren werkt, om dit te realiseren wordt er gebruik gemaakt van het perspectiefpunt.
Het detecteren van het perspectiefpunt kan gebeuren op verschillende manieren, voor dit onderzoek zijn drie verschillende technieken ge\"{i}mplementeerd, en deze zullen verder besproken worden.

De performantie en nauwkeurigheid van de 3 methoden zal besproken worden in hoofdstuk~\ref{ch:resultaten}.

\subsection{Hoogste vloerpixel segmentatie}\label{sec:seg_highest}

De eerste techniek maakt gebruik van het DeepLab-ResNet~\cite{resnet101} segmentatienetwerk gelijkaardig aan de beschrijving in hoofdstuk~\ref{sec:image_segmentation}.
Het netwerk is ge\"{i}mplementeerd in TensorFlow en voorgetraind op indoor scenes zodat het klaar is voor gebruik in python.\footnote{https://github.com/hellochick/Indoor-segmentation}

Het netwerk kan gebruikt worden om in een afbeelding elke pixel onder te verdelen in een aantal klassen zoals: muur, vloer, boom, meubel en trap.
Voor het detecteren van het perspectiefpunt hebben we gekozen om enkel de vloerpixels te gebruiken.
Het resultaat van de segmentatie is te zien in figuur~\ref{fig:floor_seg} waarbij groen vloepixels zijn, roze muurpixels, grijs plafontpixels en blauw meubelpixels.

Van de gesegmenteerde data nemen we dan de hoogste vloerpixel, en deze gebruiken we als perspectiefpunt.
\begin{figure}
    \centering
    \includegraphics[width=\linewidth]{uitwerking/segmentatie.png}
    \caption{ResNet segmentatie van 1 input frame.}
    \label{fig:floor_seg}
\end{figure}


\subsection{Vloerlijn kruising}\label{sec:hough_floor}
Voor deze techniek maken we gebruik van dezelfde segmentatiedata als in hoofdstuk~\ref{sec:seg_highest}, maar wordt er een masker gemaakt met enkel de vloerpixels.
Op dit masker wordt door middel van een Canny-edgedetector de rand van de vloer gedetecteerd, dit proces is gevisualiseerd in figuur~\ref{fig:hough_floor}.
Vervolgens kunnen er door middel van de Hough-transformatie een aantal rechten gevonden worden die raken aan zoveel mogelijk punten met de rand van de vloer.
Zoals te zien in figuur~\ref{fig:hough_floor} worden er meerdere rechten gevonden, en deze moeten gefilterd worden omdat ze ongeveer dezelfde lijn voorstellen.
Dit gebeurd door de richtingsco\"{e}ffici\"{e}nt van alle lijnen met elkaar te vergelijken, en indien het verschil te klein is, wordt de lijn met de laagste
richtingsco\"{e}ffici\"{e}nt verwijderd.
Het optimale resultaat is dat er 1 lijn overblijft aan elke kant van de vloer, zodat deze de vluchtlijnen voorstellen.
Indien dit het geval is, is de kruising van de 2 rechten het perspectiefpunt.

\begin{figure}
    \includegraphics[width=\linewidth]{uitwerking/hough_floor.png}
    \caption{\textbf{Linksboven}: vloersegmentatie, \textbf{Rechtsboven}: Canny-edgedetectie, \textbf{Linksonder}: alle Hough lijnen, \textbf{Rechtsonder}: filtering van lijnen}
    \label{fig:hough_floor}
\end{figure}


\subsection{Perspectieflijn kruising}
Deze techniek gelijkt sterk op de beschreven methode uit~\ref{sec:hough_floor} met als verschil dat hier geen gebruik gemaakt wordt van vloersegmentatie.
De gehele afbeelding wordt omgezet naar grijswaarden voordat het door een Canny-edgedetector gehaald word met een variabele treshold gebaseerd op de mediaan van de grijswaarden.
De edge detectie wordt wederom gebruikt om door middel van de Hough-transformatie een aantal aanliggende rechten te zoeken die zoveel mogelijk punten snijd.
We zijn op zoek naar perspectieflijnen, dus alle horizontale en verticale lijnen worden onmiddellijk verwijderd net als rechten waarvan de richtingsco\"{e}ffici\"{e}nt
niet genoeg van elkaar afwijkt. Dit is weergegeven in figuur~\ref{fig:hough_all}.

Indien er slechts een paar lijnen overschieten na filtering, kan gesteld worden dat het perspectiefpunt te vinden is op de kruising van de rechten.

\begin{figure}
    \includegraphics[width=\linewidth]{uitwerking/hough_all.png}
    \caption{\textbf{Links}: Canny-edgedetectie op hele afbeelding. \textbf{Rechts}: Hough lijnen uit edge detectie.}
    \label{fig:hough_all}
\end{figure}


\section{Omgevingsmeting} \label{sec:omgevingsmeting}

\begin{figure}[!h]
    \centering
    \begin{tikzpicture}[x=1cm,y=1cm,z=0.6cm,>=stealth]
        \coordinate (Camera) at (-3, 1);
        \coordinate (Obj1) at (0, 3);
        \coordinate (Obj2) at (0, 6);
        \coordinate (Obj3) at (-6, 9);

        % Side lines
        \draw (-6,0) -- (-6,12);
        \draw (0,0) -- (0,12);

        % Objects
        \node[fill,circle, inner sep=2pt, label={right:Obj}] at (Obj1) {};
        \node[fill,circle, inner sep=2pt, label={right:Obj}] at (Obj2) {};
        \node[fill,circle, inner sep=2pt, label={right:Obj}] at (Obj3) {};

        % Camera
        \node[fill,circle,inner sep=2pt, label={left:Camera}] at (Camera) {};

        % Field of view
        \draw[dashed] (-3, 1) -- (0,4);
        \draw[dashed] (-3, 1) -- (-6,4);
        \draw[dashed] (-3, 1) -- (-3,12);

        % Focal plane
        \draw[blue, name path=fplane] (-1, 3) -- (-5, 3);

        % Obj to camera
        \draw[dashed, red, name path=o2--c] (Obj2) -- (Camera);
        \draw[dashed, red, name path=o3--c] (Obj3) -- (Camera);

        \pic [draw, <->, "$\alpha_2$", angle eccentricity=1.1, angle radius=3cm] {angle = Obj2--Camera--Camera};
        \pic [draw, <->, "$\alpha_3$", angle eccentricity=1.1, angle radius=3.5cm] {angle = Camera--Camera--Obj3};

        % Projection points
        \path[name intersections={of=o2--c and fplane,by=p2}];
        \node[fill,circle,inner sep=2pt, red] at (p2) {};
        \path[name intersections={of=o3--c and fplane,by=p3}];
        \node[fill,circle,inner sep=2pt, red] at (p3) {};


        \draw[<->] ([yshift=+0.3cm]p2) -- ([yshift=+0.3cm]-3,3) node[midway, above] {$x_2$};
        \draw[<->] ([yshift=+0.4cm]p3) -- ([yshift=+0.4cm]-3,3) node[midway, above] {$x_3$};

        \draw[<->] ([xshift=+2.2cm]-3,3) -- ([xshift=+2.2cm] Camera) node[midway, left] {$l_f$};
    \end{tikzpicture}
    \caption{Schematische voorstelling omgevingsmeting}
    \label{fig:omgevingsmeting}
\end{figure}

Door middel van de \gls{rgb} camerabeelden genomen van het standpunt van de robot, willen we een representatie maken van de omgeving. Deze representatie zal later
gebruikt worden om de actuele locatie van de robot te volgen.
Voor het opbouwen van de omgeving maken we gebruik van de object detector beschreven in hoofdstuk~\ref{sec:object_detectie}.
De detector geeft een lijst terug van objecten met voor elk object een begrenzende rechthoek en een klasse.
Elk object heeft een unieke locatie in het beeld, en dus ook in de ruimte.

In figuur~\ref{fig:omgevingsmeting} is een schematische 2d voorstelling te zien van een gang waarin er zich 3 objecten bevinden, 2 van deze objecten bevinden zich
binnen de kijkhoek van de camera, en worden door middel van de rode stippellijnen geprojecteerd op het blauwe cameravlak.

De rode stippen zichtbaar in~\ref{fig:omgevingsmeting} zijn de co\"{o}rdinaten van de objecten zoals ze gedetecteerd worden door de object detector.
Het doel is om op basis van deze co\"{o}rdinaten de hoek $\alpha_i$ tussen de optische as van de camera en het object in de ruimte te bepalen.

Hiervoor komt het perspectiefpunt van pas, het perspectiefpunt licht in de figuur op een oneindige afstand op de zwarte stippellijn(optische as).
De pixelafstand volgens de x-as van de optische as tot het centrum van de bounding box van een object is gedefinieerd als $x_i$.

Formule~\ref{eq:angles} toont dat voor het verband tussen $x_i$ gemeten in pixels en $\alpha_i$ een extra constante parameter $l_f$ nodig is.
$l_f$ is \'{e}\'{e}n van de camera specifieke parameters, die de afstand tussen de camerasensor en het beeldvlak definieert, namelijk de focale lengte of brandpuntsafstand.
Deze parameter kan verkregen worden door middel van een cameraKalibratie, en moet omgerekend worden naar pixels.

\begin{equation}
    \alpha_i = \tan^{-1}(\frac{x_i}{l_f})
    \label{eq:angles}
\end{equation}


\subsection{Camera kalibratie}
Elke camera heeft een set van intrinsieke parameters die uniek zijn voor de camera/lens combinatie.
Deze parameters kunnen voorgesteld worden door middel van een cameramatrix $M$ zoals weergegeven in~\ref{eq:camera_matrix} waarbij ($f_x$, $f_y$) de focale lengtes zijn, en ($c_x$, $c_y$) de optische centra.

\begin{equation}
    M = \begin{bmatrix}
            f_x &  0  & c_x \\
            0   & f_y & c_y \\
            0   &  0  &  1
    \end{bmatrix}
    \label{eq:camera_matrix}
\end{equation}

Een eenvoudige methode om een camerakalibratie uit te voeren is een gekend patroon fotograferen uit verschillende hoeken en zo de verschillende parameters te berekenen.
Het meestgebruikte patroon voor deze toepassing is een dambordpatroon met afwisselend witte en zwarte vlakken, de raakpunten van de zwarte vlakken vormen hier een
identificeerbaar patroon. In figuur~\ref{fig:kalibratie} is een voorbeeld opgenomen van de patroondetectie.

\begin{figure}[!hb]
    \centering
    \includegraphics{uitwerking/kalibratie.png}
    \caption{Detectie van het dambordpatroon voor camerakalibratie}
    \label{fig:kalibratie}
\end{figure}


\section{Semantische kaart}\label{sec:sem_kaart}
De semantische kaart is een belangrijk onderdeel van dit werkt.
De kaart bevat een representatie van alle objecten/features die zichtbaar zijn binnen in de ruimtes.
Er is gekozen om de gegevens voor te stellen in het \gls{osm} formaat waarbij alles wordt voorgesteld door middel van node's in een XML formaat.
In figuur~\ref{fig:kaart} is een deel uit de kaart gevisualiseerd met behulp van een \gls{osm} map editor.

\begin{figure}
    \centering
    \includegraphics[width=0.5\linewidth]{uitwerking/kaart.png}
    \caption{Grafische voorstelling semantische kaart}
    \label{fig:kaart}
\end{figure}

Het \gls{osm} formaat definieert alles met behulp van node's. Elke node kan beschikken over een aantal tags waarin sleutel-waarde gegevens gelinkt kunnen worden aan de node's.
\gls{osm} is een formaat dat gebruikt wordt voor andere doeleinden dan deze toepassing, daarom hebben we gekozen om de node's op maat gemaakte tags toe te kennen
die onze eigen parser later kan interpreteren.
Aan de kaart zijn een aantal dingen toegevoegd om deze later te kunnen gebruiken.
De objecten zijn als tag toegevoegd via een map editor, en de tags zijn ingevuld in een vast patroon. Het resultaat van deze aanpassing in XML is weergegeven in listing~\ref{lst:object_node}.
De belangrijkste gegevens zijn het type en de name tag, deze geven weer dat het gaat over een 'object' met als label 'licht'.
Alle gegevens in de node tag zelf zijn ge cre\"{e}ert door de map editor bij het aanmaken van de node, waarbij 'lat' en 'lon' de wereldco\"{o}rdinaten zijn van de node.

\lstset{language=XML, caption={XML voorstelling van 1 object node}, label={lst:object_node}}
\begin{lstlisting}[basicstyle=\small]
<node id='137883' action='modify' visible='true' lat='51.068085' lon='4.500029'>
    <tag k='id' v='19' />
    <tag k='name' v='licht' />
    <tag k='type' v='object' />
</node>
\end{lstlisting}

Een 2de aanpassing aan de kaart is de beschrijving van een te volgen route in het midden van de gangen.
Een route bestaat uit een opeenvolging van node's die aan elkaar grenzen.
Een aantal van deze node's zijn toegevoegd om discrete locaties te verkrijgen die op 1 route liggen.
De XML structuur van een locatie node is zichtbaar in listing~\ref{lst:location_node}.
Een locatie node is gekenmerkt door een type='location' en bezit een aantal extra tags.

De extra tags zijn het belangrijkste onderdeel van de kaart, zij bevatten informatie over de objecten die zichtbaar zijn van op een specifieke locatie.
Zo geeft bijvoorbeeld een tag met de sleutel $21$ en waarde $10.5$ aan dat object $21$ zichtbaar is onder een hoek van ongeveer $10.5\textdegree$.

\lstset{language=XML, caption={XML voorstelling van 1 lokatie node}, label={lst:location_node}}
\begin{lstlisting}[basicstyle=\small]
<node id='137973' action='modify' visible='true' lat='51.068085' lon='4.500029'>
    <tag k='id' v='27' />
    <tag k='type' v='location' />
    <tag k='20' v='20' />
    <tag k='21' v='10.5' />
</node>
\end{lstlisting}

Een route of 'way' in \gls{osm} is een gesorteerde lijst van alle node id's die op de route liggen. Dit wordt gebruikt om het pad dat de robot zal volgen aan te geven.
In listing~\ref{lst:way} is in XML formaat een deel van de route beschreven.
De aangemaakte locatie node's zijn via de map editor gekoppeld in \'{e}\'{e}n route, deze route zal ingelezen worden door het programma, en is de enige route die de robot kan afleggen.

\lstset{language=XML, caption={XML voorstelling van een route}, label={lst:way}}
\begin{lstlisting}[basicstyle=\small]
<way id='140546' action='modify' visible='true'>
    <nd ref='4427' />
    <nd ref='137973' />
    <nd ref='137925' />
</way>
\end{lstlisting}

Het inlezen van het \gls{osm} kaartformaat gebeurd via een zelfgemaakte parser die de object, locatie en route node's inleest in een gelinkte datastructuur
zodat het eenvoudig is om later te verwerken.
De kaart kan gevisualiseerd worden met behulp van een Python \gls{osm} bibliotheek\footnote{https://github.com/gboeing/osmnx}.

\section{Lokalisatie}\label{sec:lokalisatie}