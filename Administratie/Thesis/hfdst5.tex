%%%%%%%%%%%%%%%%%%%%%%%%%%%%%%%%%%%%%%%%%%%%%%%%%%%%%%%%%%%%%%%%%%% 
%                                                                 %
%                            CHAPTER                              %
%                                                                 %
%%%%%%%%%%%%%%%%%%%%%%%%%%%%%%%%%%%%%%%%%%%%%%%%%%%%%%%%%%%%%%%%%%% 
% \chapter{Richtlijnen voor formules}

% Er zijn twee manieren om formules in LaTeX in te voeren:

% \begin{itemize}
% 	\item Inline: $a^2+b^2 = c^2$ (\verb|$a^2+b^2 = c^2$|)
% 	\item In een equation omgeving 	(\verb|\begin{equation}	a^2+b^2 = c^2	\end{equation}|):
% 	\begin{equation}
% 		a^2+b^2 = c^2
% 	\end{equation}

% \end{itemize}

% Griekse letters geef je in d.m.b. het backslash commando. Bijvoorbeeld de letter sigma $\sigma$ verkrijg je door \verb|$\sigma$| inline in te geven. Dit is analoog voor griekse letters in de equation omgeving. Een beknopte lijst van symbolen vind je op de Wikibooks pagina voor LaTeX (\href{https://nl.wikibooks.org/wiki/LaTeX/Wiskundige_formules}{link}). Alle andere nuttige informatie omtrent het gebruik van LaTeX voor formules vind je hier ook terug.
% \cleardoublepage