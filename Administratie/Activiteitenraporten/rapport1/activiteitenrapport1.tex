\documentclass[11pt,a4paper]{article}

\usepackage[a4paper,left=3.5cm, right=2.5cm, top=3.5cm, bottom=3.5cm]{geometry}
\usepackage[dutch]{babel}
\usepackage{amsmath}
\usepackage{tikz}
\usepackage{graphicx}

\setlength\parindent{0pt}                   % Fix stupid indentation on new line
\setlength\parskip{\medskipamount}

\title{Activiteitenrapport 1}
\author{Olivier Van den Eede}
\date{\today}

\begin{document}
    \maketitle
    
    \section{Reeds uitgevoerde activiteiten}
        \subsection{Object detection}
        \begin{itemize}
            \item Beeldmateriaal analyseren op zoek naar mogelijke objecten om te detecteren.
            \begin{enumerate}
                \item Hoeken
                \item Liften
                \item Rookmelder (plafont)
                \item Bordjes (pictogrammen) \label{picto}
                \item Deurklinken
                \item Deuren
                \item Vloerovergangen
                \item Stootbuizen aan de muur
                \item Brandkasten/brandblusser \label{blusser}
                \item Prikbord/magneetbord
                \item Stootpaal bij deuren
                \item Radiator
                \item Telefoon aan muur
                \item Buizen (plafont)
                \item Stopcontacten/lichtschakelaar
                \item Kast
            \end{enumerate}

            \item Voor~\ref{picto} en~\ref{blusser} zijn er een aantal experimenten gebeurd op basis van kleur tresholding, edge detection en matching via SIFT features.
            \begin{itemize}
                    \item Door de grote afwijking in kleuren tussen verschillende beelden is het zeer moeilijk om detecties te doen op basis van de kleuren.
                    \item Door de lage resolutie van het beeldmateriaal, worden er zeer weinig SIFT features gedetecteerd in de beelden.
            \end{itemize}

            \item Na overleg met de schoolpromotor over de problemen zijn er andere technieken bekeken.
            \item YOLO object detection~\cite{7780460}
            \begin{itemize}
                \item Annoteren van de dataset voor 3 klassen als experiment
                \begin{enumerate}
                    \item Exit sign
                    \item Deurklink
                    \item Brandblusser
                \end{enumerate}

                \item Hertrainen van YOLO model met nieuwe annotaties.
                \item Deze snel hertrainde versie is nog niet perfect, meer geeft wel al een goed resultaat en zal nog verder worden gebruikt.
            \end{itemize}
        \end{itemize}

        \subsection{Image segmentation}
            \begin{itemize}
                \item Exprtimenten met eenvoudige segmentatietechnieken zoals k-means.
                \begin{itemize}
                    \item Veel last vam schaduwen/overbelichting.
                \end{itemize}

                \item Experimenten met gPb segmentation~\cite{5557884}.
                \begin{itemize}
                    \item Mooie resultaten.
                    \item Zeer traag, minuten per foto.
                \end{itemize}

                \item Proberen experimenteren met SegNet image segmentation~\cite{Segnet}.
                \begin{itemize}
                    \item Niet gelukt om SegNet te gebruiken op het beeldmateriaal.
                \end{itemize}

                \item Experimenten met ander tensorflow gebaseerd Indoor-segmentatienetwerk.
                \begin{itemize}
                    \item Zonder hertraining geeft het goede resultaten op het beeldmateriaal.
                \end{itemize}
                
            \end{itemize}

    \section{Uit te voeren activiteiten}
        \begin{itemize}
            \item Op basis van alle experimenten een duidelijk beeld vormen van welke methoden er gekozen zullen worden.
            \item Deze methoden en de gemaakte keuzes neerschrijven in een literaatuurstudie.
            \item Onderzoek over object tracking met YOLO.
            \item Onderzoek hoe actuele locatie gebruikt kan worden om resultaten te verbeteren.
            \item Onderzoek over de koppeling met een semantische kaart.
        \end{itemize}

    \bibliographystyle{plain}
    \bibliography{literatuur.bib}

\end{document}